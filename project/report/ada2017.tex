%
% File acl2014.tex
%
% Contact: giovanni.colavizza@epfl.ch
%%
%% Based on the style files for ACL-2013, which were, in turn,
%% Based on the style files for ACL-2012, which were, in turn,
%% based on the style files for ACL-2011, which were, in turn, 
%% based on the style files for ACL-2010, which were, in turn, 
%% based on the style files for ACL-IJCNLP-2009, which were, in turn,
%% based on the style files for EACL-2009 and IJCNLP-2008...

%% Based on the style files for EACL 2006 by 
%%e.agirre@ehu.es or Sergi.Balari@uab.es
%% and that of ACL 08 by Joakim Nivre and Noah Smith

\documentclass[11pt]{article}
\usepackage{acl2014}
\usepackage{times}
\usepackage{url}
\usepackage{latexsym}

%\setlength\titlebox{5cm}

% You can expand the titlebox if you need extra space
% to show all the authors. Please do not make the titlebox
% smaller than 5cm (the original size); we will check this
% in the camera-ready version and ask you to change it back.


\title{Name of your project}

\author{First Author \\
  {\tt email@domain} \\\And
  Second Author \\
  {\tt email@domain} \\\And
Third Author \\
{\tt email@domain} \\}

\date{}

\begin{document}
\maketitle
\begin{abstract}
  Social networks now have a huge importance in our lives and many people use them to comment about events that are happening around the globe. With this project, we would like to see how the Swiss Twitter community reacts to important events happening in Switzerland or around the world. Our main goal is to determine to what extent and how well we can learn about what is happening in the world or in our country based on the Swiss tweets. The story we want to tell is the evolution of tweets during important events between 2010 and 2016 and discover what kind of events Swiss people are tweeting about the most. We are motivated to do this project and tell this story because none of us are active on Twitter and we're interested in understanding better how twitter is used in Switzerland.
\end{abstract}

\section{Credits}


\section{Introduction}


\section{Pre-processing}

At first, we extract the hashtags from each tweets. Each tweet is stored in a dataframe that only contains the informations we require, namely: 
\begin{itemize}

\item The tweet id 
\item The user id
\item The longitude and latitude
\item The hashtags extracted from the text
\item The day, month and year of the tweet creation

\end{itemize}

After the hashtag extraction, we only keep the tweets with at least one hashtag.


\section{Data Manipulation}

\subsection{Grouping by hashtag}

\section{Data analysis and visualization}

\section{Event detection}


\begin{thebibliography}{}

\bibitem[\protect\citename{Gusfield}1997]{Gusfield:97}
Dan Gusfield.
\newblock 1997.
\newblock {\em Algorithms on Strings, Trees and Sequences}.
\newblock Cambridge University Press, Cambridge, UK.

\end{thebibliography}

\end{document}
